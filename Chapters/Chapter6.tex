%Chapter 6 - Re-evaluating the MEGA65 project against known risks
% 	Did the MEGA65 project reduce its risk after 10 months?

\chapter{Risk evaluation of the MEGA65 project in May 2019}
\label{Chapter6}
After a 10 month period, the MEGA65 project was re-evaluated against the risks identified in Chapter \ref{Chapter3}. This period of time gave the project sufficient time to act on the suggestions given at the end of Chapter \ref{Chapter5}. By re-evaluating the MEGA65 project and comparing the result with the evaluation from July 2019, it is hoped that a conclusion can be drawn on whether the MEGA65 project reduced its risk exposure or not. 
%----------------------------------------------------------------------------------------
%----------------------------------------------------------------------------------------

\section{Laws and regulations}
The MEGA65 project has undertaken some research into the laws and regulation which need to be upheld in some of their key markets in which they expect to operate. With the knowledge gained from the research, the uncertainty in the relevant laws and regulations is reduced, as such the overall risk to the project is reduced. Not every market was researched, and the unknown markets still expose the project to risk if the MEGA65 where to be sold within it. The MEGA65 project is rated as having an unlikely chance of occurring and if it causing a marginal level of potential damage. \\

\begin{tabular}{l|l} % <-- Alignments: l=left, c=centre, r=right
    	\textbf{Category} 	&	\textbf{Rating} \\
      \hline
     Likelihood			&	Unlikely \\
     Possible Damage 	& 	Marginal \\
     Risk 				&	Low		\\	
    \end{tabular}


\section{Sourcing old components}
During the interim 10 months the MEGA65 project purchased 700 3.5" floppy disk drives and has them in storage in Germany. These disk drives are to be used as components during the assembly of the MEGA65 desktop version. This action of purchasing a large surplus of stock allows the MEGA65 project to avoid this risk in the medium term. When the stock runs out the risk will increase, as such the risk is largely tied to the MEGA65's popularity and sales volume. If the MEGA65 becomes hugely popular then this risk could increase dramatically. If the MEGA65 sells as expected by the MEGA65 project team, then 700 drives will be sufficient for the medium to long term. The potential damage the MEGA65 project is rated as marginal and the chance of it occurring possible. \\

\begin{tabular}{l|l} % <-- Alignments: l=left, c=centre, r=right
    	\textbf{Category} 	&	\textbf{Rating} \\
      \hline
     Likelihood			&	Possible \\
     Possible Damage 	& 	Marginal \\
     Risk 				&	Moderate		\\	
    \end{tabular}


\section{Use of 3rd party intellectual property}
The MEGA65 project has enacted some of the suggestions from Chapter \ref{Chapter5}. The steps undertaken in the identified risk areas are discussed below.

\subsection{Commodore logo}
The MEGA65 team decided to use their own design for the logo on the Commodore keyboard key. They have created and used the design for the pre-production run of the desktop form factor. This new design should completely remove the potential for this risk to effect the MEGA65 project.

\subsection{Commodore 64 character set}
The MEGA65 have decided to replace the Commodore character set with one they have created themselves from other freely available character sets and their own work. By removing the Commodore character set the risk is eliminated. 

\subsection{Commodore BASIC ROM and kernel ROM}
The MEGA65 team have decided to created their own BASIC and kernel ROMs. This reimplementation is currently under way and can provide basic functionality at this stage. By removing the Commodore ROMs from the MEGA65, the risk is entirely eliminated. \\

\begin{tabular}{l|l} % <-- Alignments: l=left, c=centre, r=right
    	\textbf{Category} 	&	\textbf{Rating} \\
      \hline
     Likelihood			&	Rare \\
     Possible Damage 	& 	Negligible \\
     Risk 				&	Low		\\	
    \end{tabular}


\section{Open-source business}
The MEGA65 team have not trademarked the MEGA65 name and are operating under the assumption that the MEGA65 will not be hugely popular. The MEGA65 team have been made aware of the strategies listed in Chapter \ref{Chapter5}, these strategies are only sensible if the MEGA65 becomes hugely popular and cannot be enacted to much benefit before. Because the MEGA65 team is aware of these strategies the MEGA65 project's potential damage and chance of occurring are both lowered compared to July 2018. \\

\begin{tabular}{l|l} % <-- Alignments: l=left, c=centre, r=right
    	\textbf{Category} 	&	\textbf{Rating} \\
      \hline
     Likelihood			&	Unlikely \\
     Possible Damage 	& 	Moderate \\
     Risk 				&	Low		\\	
    \end{tabular}


