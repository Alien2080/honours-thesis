%Chapter 3 - Bringing open-source 8-bit projects to the market: challenges
%What are the challenges and pitfalls that face productising open-source retro computer projects?

\chapter{Bringing open-source 8-bit projects to the market: Case studies of similar projects and the challenges they faced}
\label{Chapter3}
By considering other similar projects and and highlighting the challenges they faced it is hoped that the MEGA65 project can avoid unnecessary delays and costs in the productisation process. This chapter outlines challenges that where faced by the teams behind the Spectrum Vega, Spectrum Vega+, Spectrum Next, Raspberry Pi, Arduino, C64DTV and C64Mini. These challenges where uncovered during case studies into these projects. This chapter also strives to highlight methods and activities used by the projects studied, that will help ensure the success of the MEGA65 project. 

%----------------------------------------------------------------------------------------
%----------------------------------------------------------------------------------------
\section{Open-source Case Studies}
The Arduino and Raspberry Pi where both studied because of their open-source nature, success and because both are relatively simple computers (in the general sense). These project share enough characteristics with the MEGA65 that some useful information was obtainable from these studies.

\subsection{Arduino}
\textbf{What is it?}
Arduino is a company which designs, produces and sells the Arduino range of single-board microcontrollers, which are mostly 8-bit machines with 32 bit machines being introduced later. The focus of this case study is the first Arduino board released for general sale to the public and the process to get to that point. The Arduino company has released all the hardware designs as open-source under a Creative Commons licence. The associated IDE software is also open-source and Arduino also encourage and facilitate a community of hobbyist and open-source enthusiasts. 

\textbf{When was it produced?}
The first Arduino board was designed and made available to students at the Interaction Design Institute Ivrea (IDII) in 2005. It was used to help teach the students interactive design (also known as physical computing). The board quickly became popular outside of the class and Arduio boards remains hugely popular today.

\textbf{Why was it produced?}
To help students at the IDII. The Arduino creator was using another commercially available microcontroller before but it didn't meet his needs for teaching it was too expensive. It drew heavy inspiration from another very similar project, Wiring, that was also in development at IDII before and during the creation of the first Arduino board. Wiring in turn built on another related project, Processes, which provided the IDE used in Wiring (with some modifications). 

\textbf{Challenges faced during productisation of first Arduino microcontroller}
The biggest challenge for the Arduino company, in my opinion, was simultaneously making a profit for the company while being open-source with their products and allowing anyone to manufacture and sell them. Arduino handled this very successfully, with both the designs still being open-source and the company (Arduino LLC) still trading and appears to be reasonably profitable but as it is a private company it doesn't need to publish earnings. The original decision to make the designs open-source, stemmed from the realisation the IDII was running out of money and would shut soon, potential endangering the Arduino project. The Arduino team partnered with a manufacturer and got the first boards produced and sold for a small profit. Arduino found that as they were the first to market (with good reason) that they still sold a lot due to lack of competition. But when cheaper copies started being manufactured in Taiwan and China, this initially had the effect of increasing the sales through Arduino, even though their where cheaper version available. Arduino put this down to the competition being of a lesser quality in the beginning. Arduino also charged a fee if other producers wanted to use the "Arduino" name. As the market matured, Arduino expected other producers and manufactures to be able to produce the Arduino boards at a cheaper price and of similar quality, which has happened. The Arduino company focuses more on selling their expertise as the inventors, consultancy work, enabling the Arduino community and designing and selling Arduino accessories, such as "shields" that add extra functionality to the Arduino range.

The other big challenge Arduino faced was from losing control of their Intellectual Property (IP). Arduino originally decided to only register their trademark in the U.S.A. 