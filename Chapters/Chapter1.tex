% Chapter 1 - Introduction
\chapter{Introduction} % Main chapter title
\label{Chapter1} % For referencing the chapter elsewhere, use \ref{Chapter1} 
%Copied from Thesis Topic Manual
%Introduction
%The introduction should provide the following:
% background to the topic;
% brief review of current knowledge (this is not the literature review – it’s only a high level overview);
% state hypotheses;
% indicate gaps in knowledge, state aims of the thesis and how it fits into the gap;
% an outline of the following chapters.

%The introduction should follow the recommended structure:
% state the general background of the thesis topic and give some background
% provide an overview of the literature related to the thesis topic
% define the terms and scope of the thesis
% outline the current situation
% evaluate the advantages/ disadvantages of existing solutions and identify the knowledge gap
% identify the importance of the proposed research
% state the research problem/ questions
% state the research aims and/or research objectives
% state the hypotheses
% outline the experiment methodology
% outline the structure of the thesis
%----------------------------------------------------------------------------------------
%----------------------------------------------------------------------------------------
A retro computing or a retro revival project is a computing project that is in some way inspired or derived from a home computer the was released between 1975 and 1990. This period is referred to as the home computer era and it was the first time computers where specifically marketed for personal use within the home. This lead to many people first interacting with computers during this period. For various reasons, many people have tried, with varying amounts of success, to revive certain computer systems from the period, either by emulating the system in software on a more powerful computer or rebuilding the circuitry. Other projects have been inspired by the home computers of the period being less complex and with less abstraction between the user and the mechanics of the computer. The MEGA65 is a retro computing project which aims to innovate the unreleased Commodore 65, it is currently undergoing development and has not released a product as of yet. 

\section{Home computer era (provide overview of literature)}


\section{Scope of thesis}
This thesis aims to provide a body of knowledge on the process as well as the challenges and risks associated with a retro computing project.

\section{Current situation}
Many retro computing projects have been completed with varying amounts of success. The widely varying processes and methods employed by differing projects, suggests the outcomes may be more consistent and improved if a more rigorous process was followed. Another way to improve the outcomes of retro revival projects is to highlight the common challenges and risks that retro projects are likely to be exposed to, which should allow retro projects to be evaluated against aforementioned risks. With a risk evaluation undertaken and the high risk areas identified, the retro project should be able to determine and enact strategies to reduce their risk exposure, this should then have the effect of improving the outcomes of the project.

\section{Research questions}
The questions this thesis aims to answer are:
\begin{enumerate}
\item What does the retro computing revival project productization process look like?
\item What risk and challenges are associated with a retro computing revival project?
\item What is the MEGA65 project? 
\item How exposed to risks is the MEGA65 project July 2018?
\item What can be done to reduce the MEGA65 project's risks?
\item Did the MEGA65 project reduce its risk after 10 months?
\end{enumerate}

\section{Hypothesis}
It is hypothesised that the body of knowledge created, which includes a definition of the process to productization of a retro computing project as well as its associated risks and challenges, will allow future retro computing projects achieve more desirable outcomes more consistently.

\section{Methodology}
To answer the research questions and produce a body of knowledge, an in depth case study is conducted into several retro computing projects. From these case studies a process is distilled and recorded. The challenges that beset each project in the case study are recorded and categorised into risks. As a way to establish the quality of the risks ascertained, a retro computing project is evaluated twice against the aforementioned risks. This project, called MEGA65, is currently in development, as such it is a good candidate to evaluate. The two evaluation take place ten months apart, after the first evaluation the MEGA65 project is provided with advice or strategies on how to reduce their risk in the high risk areas. After ten months the MEGA65 project is evaluated again and the evaluations are compared to determine if their risk expose has changed. To evaluate the MEGA65 project effectively, it first needed to be understood. Chapter 4 aimed to detail the MEGA65 project at a snapshot in time, which was July 2018. To conduct this research into the MEGA65, due to the nature of the project, it was mostly conducted with conversations with the team members. 

%----------------------------------------------------------------------------------------
%----------------------------------------------------------------------------------------
\section{Structure of thesis}
