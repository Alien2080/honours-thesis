%Chapter 7 - Conclusion and duture directions
%What is required to bring the MEGA65 to a MVP and market?

\chapter{Conclusion and future direction}
\label{Chapter7}
This chapter discusses the results obtained in Chapter \ref{Chapter6} and reaches a conclusion on the hypothesis stated in Chapter \ref{Chapter1}. Directions for future research are also discussed.

\section{Discussion of results}
The results from chapter 6 show that the risk exposure for the MEGA65 project decreased over the ten month period between the first and second risk evaluation, table \ref{tab:table1}. This reduction was caused by the MEGA65 team undertaking actions to mitigate the risks. These actions where spurred by the advice given to the MEGA65 team after the first risk evaluation. Directly proving how effective the risk evaluation and subsequent advice were in improving the outcomes for the MEGA65 project is difficult, as it is a sample size of one and there is no practical way to have a control group to study. Also, the ten month time frame made it impossible to determine the final outcomes of the project, as such the value of the results is lessened.

\section{Conclusion}
As mentioned above, the results from the comparison of the risk evaluations show that the MEGA65 project has reduced its risk exposure in the interim ten months, table \ref{tab:table1}. This reduction in risk exposure is an improved outcome for the MEGA65 project. This improved outcome is partly due to the body of knowledge within this thesis, and specifically the identified retro-computing risks and the risk evaluation of the MEGA65, based on these risks. This improved outcome for the MEGA65 retro-computing project proves the hypothesis stated in Chapter \ref{Chapter1}.

\section{Future direction of research}
A possible future direction of research is further testing of the hypothesis, by undertaking another study of the MEGA65 at the end of the productization process, to determine if the MEGA65 did suffer from any of the events described by the risks identified in the case studies. This body of knowledge could also be given to other retro-computing projects at the beginning of their productization process. The projects could then be studied to determine if the body of knowledge provides any benefit to the outcomes of these projects.

Another direction that research could continue is to further elaborate on the body of knowledge contained within this thesis. As there was no prior work done in this area, the aim was to cover a broad area with as much depth as was possible within the time frame. This leaves a lot of areas within that could  be potential areas of future research. The identified risks could be elaborated into many more risk categorises with much more description. The process identified in the case study could also be elaborated with each step offering a possible area of research.
