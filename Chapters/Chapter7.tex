%Chapter 7 - Results and discussions

\chapter{Results and discussions} 
\label{Chapter7}
This chapter is dedicated to discussing the results of this thesis, specifically the comparison of risk evaluations conducted on the MEGA65 project, how the thesis answered the research questions and a consideration of the hypothesis. 

%%%%%%%%%%%%%%%%%%%%%%%%%%%%%%%%%%%%%%%%%%%%%%%%%%%%%%%%%%%%%%%%%%%%%%%%%%%%%%%%%%%%%%
\section{Comparing risk evaluations from July 2018 with May 2019}
\label{comparing risk evaluations}
By comparing the risk evaluation of the MEGA65 project from July 2018 with the evaluation from May 2019 it can be seen how the MEGA65 project's risk exposure changed during that time. Table \ref{tab:table1} shows the risk evaluation results for the risk areas in which the MEGA65's risk exposure changed between July 2018 and May 2019.

It can be seen that 7 of the total 17 risks identified in chapter \ref{Chapter5} have been reduced.

\begin{table}[h!]
  \begin{center}
    \caption{Comparing risk evaluations from July '18 and May '19}
    \label{tab:table1}
    \begin{tabular}{l|l|l|l} % <-- Alignments: l=left, c=centre, r=right
    	\textbf{Risk} 	&	\textbf{July 2018} & \textbf{May 2019} & \textbf{Change}\\
      \hline
     Laws and regulations 				& High		& Low 		& Reduced \\
     Sourcing old components			& High		& Moderate	& Reduced \\
     Commodore logo						& Extreme	& Low  		& Reduced \\
     Commodore character set			& High		& Low  		& Reduced \\
     Commodore BASIC ROM				& High		& Low 		& Reduced \\
     Commodore kernel ROM				& High		& Low  		& Reduced \\
     Open-source business				& High		& Low  		& Reduced \\
    \end{tabular}
  \end{center}
\end{table}


%%%%%%%%%%%%%%%%%%%%%%%%%%%%%%%%%%%%%%%%%%%%%%%%%%%%%%%%%%%%%%%%%%%%%%%%%%%%%%%%%%%%%%
\section{Discussion of results}
The results from table \ref{tab:table1} show that the risk exposure for the MEGA65 project decreased over the ten month period between the first and second risk evaluation. This reduction was caused by the MEGA65 team undertaking actions to mitigate the risks. These actions where spurred by the advice given to the MEGA65 team after the first risk evaluation. 

Directly proving how effective the risk evaluation and subsequent advice were in improving the outcomes for the MEGA65 project is difficult, as it is a sample size of one and there is no practical way to have a control group to study. Also, the ten month time frame made it impossible to determine the final outcomes of the project, as such the value of the results is lessened.

%%%%%%%%%%%%%%%%%%%%%%%%%%%%%%%%%%%%%%%%%%%%%%%%%%%%%%%%%%%%%%%%%%%%%%%%%%%%%%%%%%%%%%
\section{Research questions}
This section considers each research question in turn. For each question, an answer and a discussion of how the answer was achieved, is given.

\subsection{What does the retro-computing project productization process look like?}
The retro-computing process is discussed and shown in section \ref{sec: Retro-computing process}. This process was elicited from the case studies of retro-computing projects seen in chapter \ref{Chapter3}. 

\subsection{What risk and challenges are associated with a retro computing project?}
The risk and challenges are discussed in section \ref{sec: Retro-computing risks}. The challenges are obtained from the case studies into retro-computing projects, shown in chapter \ref{Chapter3}. The risk are then derived from these challenges and shown in figure \ref{risk list}.

\subsection{What is the MEGA65 project?}
The MEGA65 project is discussed in detail in chapter \ref{Chapter4}. The research into the MEGA65 project was largely conducted via conversations with the team members due to the nature of the productization process limiting the amount of documentation generated. A use case study was conducted on the MEGA65 to determine the requirements needed to fulfil the use cases. From these requirements, a snapshot of the state of the MEGA65 as of July 2018 was obtained.

\subsection{How exposed to risks was the MEGA65 project in July 2018?}
A risk evaluation of the MEGA65 project, against the identified risks, was conducted in section \ref{evaluation18}. This evaluation directly answers this question. 

\subsection{What can be done to reduce the MEGA65 project's risks?}
Advice on how the MEGA65 project can reduce its risk exposure was given in section \ref{recommendations}. This advice was arrived at by considering the factors which affect each risk. These factors are outlined in section \ref{sec: Retro-computing risks}. For each factor, possible methods to positivity affect it in regards to risk where determined.

\subsection{Did the MEGA65 project reduce its risk after 10 months?}
A comparison of the two risk evaluations of the MEGA65 project show that the project did reduce its risk over 10 months, table \ref{tab:table1}. The second evaluation focused on the risk areas which where identified as high risk within the first risk evaluation. 

%%%%%%%%%%%%%%%%%%%%%%%%%%%%%%%%%%%%%%%%%%%%%%%%%%%%%%%%%%%%%%%%%%%%%%%%%%%%%%%%%%%%%%
\section{Hypothesis}
This section considers the hypothesis stated at the beginning of this thesis in section \ref{hypothesis}. The hypothesis states that outcomes to retro-computing projects will be improved with the creation of a body of knowledge devoted to the retro-computing productization process. To prove this hypothesis the retro-computing project, MEGA65 was studied. 

A risk evaluation of the MEGA65 project in July 2018 was conducted. This evaluation focused on the risks identified in the retro-computing case studies. From this evaluation the high risk areas where identified, and advice given on how to mitigate these risks. A second risk evaluation was conducted 10 months later and the results of both compared.

The advice drew from the body of knowledge contained in the case studies and their resulting process and risk discussions contained in chapter \ref{Chapter3}. The risks with which the MEGA65 was evaluated against where directly resultant from the body of knowledge. 

Considering the MEGA65 project had a reduced risk exposure as a result of the advice given in this thesis. This reduced risk exposure is an improved outcome for the MEGA65 project. And considering the advice and risks where both associated with the body of knowledge on retro-computing projects contained in \ref{Chapter3}, the hypothesis can be considered correct.