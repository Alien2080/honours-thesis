%Chapter 4 - MEGA65: A retro revival project
%What does the retro computer productization process look like?

\chapter{MEGA65: A retro revival project}
\label{Chapter4}

This chapter outlines the current state of retro revival project, the MEGA65, mentioned in \ref{Chapter1}. By first giving a description of the intended products of the project as the project members intend. Then a use case study of the MEGA65 is carried out to further describe the product, it's intended markets and to elicit requirements to realise the use cases. 
%----------------------------------------------------------------------------------------
%----------------------------------------------------------------------------------------
\section{MEGA65 state at beginning of thesis}
The MEGA65 is a innovation of the Commodore 65 built using FPGA technology at its core, more details are given in section \ref{History of the MEGA65 project}. The MEGA65 is planned to have two separate form-factors: a hand-held game console with 2 cellular modem sockets to allow for telephony, called MEGAphone and a desktop form-factor inspired by the look of the Commodore 64. During development of the MEGA65, the MEGA65 team got into discussions with the Museum of Electronic Games and Art (MEGA)and came to an agreement with MEGA that they will handle the physical development of the MEGA65 desktop version. Both form-factors are to have the same FPGA chip and FPGA implementation. This means they are the same computer at the core and can run the same software, in fact both form-factors are intended to ship with the same software, both custom software made specifically for the MEGA65 and 3rd party software such as the Commodore 65 and 64 ROMs which store the versions of BASIC needed to run specific modes of the MEGA65. These modes allow for compatibility with different software designed for the different versions of BASIC. The MEGAphone is to be developed further at Flinders University, with the original MEGA65 team members as well as involving students from the university. The desktop version is to have its physical development handled by MEGA. Physical 
