\documentclass[10pt,a4paper]{report}
\usepackage[utf8]{inputenc}
\usepackage{amsmath}
\usepackage{amsfonts}
\usepackage{amssymb}
\begin{document}
\chapter{Literature Review}
\section{Early home computer era}
The early home computer era started about mid-1970 and went until about late-1980s. This era was made possible by earlier technological advances such the transistor replacing vacuum tubes, and the invention of the Integrated Circuit (IC, sometimes referred to as a microchip). These advances allowed smaller and cheaper computers to be built, sometimes referred to as microcomputers. In the 70s, intel also developed the microprocessor; A microprocessor is a CPU incorporated within a single IC or several ICs. The first Microprocessor was the 4-bit Intel 4004, released in 1974.

For the first time it made business sense to market computers to personal users for their homes (hence the name home computer era). The lead to a exponential growth in the number of home computers. In this period, there was a general feeling and consensus among the western world, that this technological revolution was going to be extremely important in the future. This view was at least partial encouraged by the advertisements and marketing arms of the various computer companies that sprung up around this time (due to the technical advances). Many parents and education institutions at the time acclaimed the benefits of computers and learning computer programming. These factors encouraged a boom in home computers and changed the lives of many in the process.

Some of the more popular computers from this period used the MOS 6502 microprocessor or a derivation of it. The 6502 is an 8-bit CPU/microprocessor that was vastly cheaper than the alternatives at the time of its creation. 1974 saw the first commercially successful home computer: the Altair 880, this used an Intel 8080 microprocessor. 


\section{Commodore 64, 128, 65}

\section{History of the MEGA65 project}

\section{The progressing complexity of computer hardware and software}

\section{History of computer insecurity}

  

\end{document}