%Chapter 8 - Conclusion and duture directions

\chapter{Conclusion and future direction of research}
\label{Chapter8}
This chapter provides a conclusion and directions for future research.

\section{Conclusion}
As mentioned in section \ref{comparing risk evaluations}, the results from the comparison of the risk evaluations show that the MEGA65 project has reduced its risk exposure in the interim ten months, table \ref{tab:table1}. This reduction in risk exposure is an improved outcome for the MEGA65 project. This improved outcome is partly due to the body of knowledge within this thesis, and specifically the identified retro-computing risks and the risk evaluation of the MEGA65, based on these risks. This improved outcome for the MEGA65 retro-computing project proves the hypothesis stated in Chapter \ref{Chapter1}.

As stated in section \ref{contributions}, the contributions this thesis makes to retro-computing projects is categorised as follows:
\begin{enumerate}
\item Creates a body of knowledge exploring the retro-computing project productization process.
\item Creates a body of knowledge discussing the risks and challenges associated with retro-computing projects.
\item A risk evaluation of a specific retro-computing project, the MEGA65.
\item Practical, actionable recommendations to reduce the MEGA65 project's risk exposure.
\item Evidence of the benefit of those recommendations, gathered through a follow-up risk evaluation of the MEGA65 several months after providing the recommendations.
\end{enumerate}

\section{Future direction of research}
A possible future direction of research is further testing of the hypothesis, by undertaking another study of the MEGA65 at the end of the productization process, to determine if the MEGA65 did suffer from any of the events described by the risks identified in the case studies. This body of knowledge could also be given to other retro-computing projects at the beginning of their productization process. The projects could then be studied to determine if the body of knowledge provides any benefit to the outcomes of these projects.

Another direction that research could continue is to further elaborate on the body of knowledge contained within this thesis. As there was no prior work done in this area, the aim was to cover a broad area with as much depth as was possible within the time frame. This leaves a lot of areas within that could  be potential areas of future research. The identified risks could be elaborated into many more risk categorises with much more description. The process identified in the case study could also be elaborated with each step offering a possible area of research.
